\chapter{基础模块示例}

\section{特殊文本类型}
\subsection{脚注}
% 如果你的项目来源于科研项目,可以使用以下指令插入无编号脚注于正文第一页
\blfootnote{本项目来源于科研项目“基于\LaTeX{}的本科毕业设计”,项目编号1124}
社交媒体是一种供用户创建在线社群来分享信息、观点、个人信息和其它内容(如视频)的电子化交流平台,社交网络服务(social network service, SNS)和微博客(microblogging)都属于社交媒体的范畴\cite{webster_social_media},国外较为知名的有Facebook\footnote{http://www.facebook.com/}、Instagram\footnote{https://www.instagram.com/}、Twitter\footnote{http://www.twitter.com/}、LinkedIn\footnote{http://www.linkedin.com/}等,国内较为知名的有新浪微博\footnote{http://www.weibo.com/}。

在社交媒体的强覆盖下,新闻信息的传播渠道也悄然了发生变化。\cite{false_news_spread_2018}

\subsection{定义、定理与引理等}
\begin{definition}
这是一条我也不知道在说什么的定义,反正我就是写在这里做个样子罢了,也没人会仔细读。\cite{周兴2017基于深度学习的谣言检测及模式挖掘}
\end{definition}

\begin{theorem}
这是一条我也不知道在说什么的定理,反正我就是写在这里做个样子罢了,也没人会仔细读。
\end{theorem}

\begin{axiom}
这是一条我也不知道在说什么的公理,反正我就是写在这里做个样子罢了,也没人会仔细读。
\end{axiom}

\begin{lemma}
这是一条我也不知道在说什么的引理,反正我就是写在这里做个样子罢了,也没人会仔细读。
\end{lemma}

\begin{proposition}
这是一条我也不知道在说什么的命题,反正我就是写在这里做个样子罢了,也没人会仔细读。
\end{proposition}

\begin{corollary}
这是一条我也不知道在说什么的推论,反正我就是写在这里做个样子罢了,也没人会仔细读。
\end{corollary}

\subsection{中英文文献、学位论文引用}
根据美国皮尤研究中心的2017年9月发布的调查结果\cite{pew_news_use_2017},67\%的美国民众会从社交媒体上获取新闻信息,其中高使用频率用户占20\%。在国内,中国互联网信息中心《2016年中国互联网新闻市场研究报告》\cite{internet_news_2016}也显示,社交媒体已逐渐成为新闻获取、评论、转发、跳转的重要渠道,在2016年下半年,曾经通过社交媒体获取过新闻资讯的用户比例高达90.7\%,在微信、微博等社交媒体参与新闻评论的比例分别为62.8\%和50.2\%。社交媒体正在成为网络上热门事件生成并发酵的源头,在形成传播影响力后带动传统媒体跟进报道,最终形成更大规模的舆论浪潮。\cite{Yang2012Automatic}

在国内,新浪微博由于其发布方便、传播迅速、受众广泛且总量大的特点,成为了虚假信息传播的重灾区:《中国新媒体发展报告(2013)》\cite{唐绪军2013中国新媒体发展报告}显示,2012年的100件微博热点舆情案例中,有超过1/3出现谣言;《中国新媒体发展报告(2015)》\cite{唐绪军2015中国新媒体发展报告}对2014年传播较广、比较典型的92条假新闻进行了多维度分析,发现有59\%的虚假新闻首发于新浪微博。

此等信息的传播严重损害了有关公众人物的名誉权,降低了社交媒体服务商的商业美誉度,扰乱了网络空间秩序,冲击着网民的认知,极易对民众造成误导,带来诸多麻烦和经济损失,甚至会导致社会秩序的混乱。针对社交媒体谣言采取行动成为了有关部门、服务提供商和广大民众的共同选择。\cite{周兴2017基于深度学习的谣言检测及模式挖掘}

\section{图表及其引用}
此处引用了简单的表\ref{crowdwisdom_TMP}。

请注意,\LaTeX{}的图表排版规则决定了图表\textbf{不一定会乖乖呆在你插入的地方},这是为了避免Word中由于图片尺寸不匹配在页面下部出现的的空白,所以请不要使用“下图”“下表”作为指向文字,应使用“图1-1所示”这样的表述。

\begin{bupttable}{基于浏览者行为的特征}{crowdwisdom_TMP}

    \begin{tabular}{l|l|l}
		\hline \textbf{特征} & \textbf{描述} & \textbf{形式与理论范围}\\
		\hline 点赞量 & 微博的点赞数量 & 数值,$\mathbb{N}$ \\
		\hline 评论量 & 微博的评论数量 & 数值,$\mathbb{N}$ \\
		\hline 转发量 & 微博的转发数量 & 数值,$\mathbb{N}$ \\
		\hline
    \end{tabular}
\end{bupttable}

此处引用了复杂的表\ref{complexcrowdwisdom_TMP}。


\begin{bupttable}{基于浏览者行为的复杂特征}{complexcrowdwisdom_TMP}
    \begin{tabular}{l|l|l|l}
        \hline
        \multicolumn{1}{c|}{\multirow{2}{*}{\textbf{类别}}} & \multicolumn{1}{c|}{\multirow{2}{*}{\textbf{特征}}} & \multicolumn{2}{c}{\textbf{不知道叫什么的表头}} \\
        \cline{3-4}
        & & \multicolumn{1}{c|}{\textbf{描述}} & \multicolumn{1}{c}{\textbf{形式与理论范围}} \\
        \hline
        \multirow{3}{*}{正常互动} & 点赞量 & 微博的点赞数量 & 数值,$\mathbb{N}$ \\
        \cline{2-4}
        & 评论量 & 微博的评论数量 & 数值,$\mathbb{N}$ \\
        \cline{2-4}
        & 转发量 & 微博的转发数量 & 数值,$\mathbb{N}$ \\
        \hline
        非正常互动 & 羡慕量 & 微博的羡慕数量 & 数值,$\mathbb{N}$ \\
        \hline
    \end{tabular}
\end{bupttable}

此处展示了更专业的表\ref{tab:abbr_table},一个好的表格没有竖线。
% 请注意1)tabularx环境对多行文本的处理;2)booktabs宏包中支持的更粗的顶端和底端表格边界线,边界线与文本间更大的间距。
\begin{bupttable}{红警2名词解释}{tab:abbr_table}
    \begin{tabularx}{\textwidth}{llX}
        \toprule
        \textbf{术语类别} & \textbf{缩略语} & \textbf{解释} \\ \midrule
        & 兵营 & 兵营(Barracks),《命令与征服\ 红色警戒2:尤里的复仇》游戏中的一种生产建筑,用以生产步兵单位 \\ \cmidrule(l){2-3}
        & 建造场 & 建造场(Construction Yard),《命令与征服\ 红色警戒2:尤里的复仇》游戏中的一种基础建筑,用以支持其他建筑的建造 \\ \cmidrule(l){2-3}
        & 矿厂 & 矿石精炼厂(Ore Refinery),《命令与征服\ 红色警戒2:尤里的复仇》游戏中的一种资源建筑,用以将矿车采集的矿石转化为游戏资金 \\ \cmidrule(l){2-3}
        游戏 & 空指 & 空指部(Airforce Command Headquarters),《命令与征服\ 红色警戒2:尤里的复仇》游戏中的一种资源建筑,用以提供雷达功能和T2科技及生产部分空军单位 \\ \cmidrule(l){2-3}
        & 相机 & 游戏术语,特指游戏内的观察区域和视角 \\ \cmidrule(l){2-3}
        & 重工 & 战车工厂(War Factory),《命令与征服\ 红色警戒2:尤里的复仇》游戏中的一种生产建筑,用以生产载具单位 \\ \cmidrule(l){2-3}
        & 战争迷雾 & 游戏术语,《命令与征服\ 红色警戒2:尤里的复仇》中指黑色的未探索区域 \\ \bottomrule
    \end{tabularx}
\end{bupttable}

此处引用了一张图。图\ref{autoencoder_TMP}表示的是一个由含有4个神经元的输入层、含有3个神经元的隐藏层和含有4个神经元的输出层组成的自编码器,$+1$代表偏置项。

%图片宽度设置为文本宽度的75%,可以调整为合适的比例
\buptfigure[width=0.7\textwidth]{pictures/autoencoder}{自编码器结构}{autoencoder_TMP}

%组图示例,已按照指导手册要求设计,由于子图数量不同,无法压缩成\buptfigure那样,大家对照示例即可
\begin{figure}[!htbp]
    \centering
    \subfloat[]{ %[]对齐方式,t为top,b为bottom,留空即可
	\label{Fig:R1} % 子图1标签名
    	\includegraphics[width=0.45\textwidth]{pictures/autoencoder} %插入图片命令,格式为[配置]{图片路径}
    }
    \quad %空格
    \subfloat[]{
	\label{Fig:R2} % 子图2标签名
    	\includegraphics[width=0.45\textwidth]{pictures/autoencoder}
    }
    \caption{这是两个自编码器结构,我就是排一下子图的效果:\protect\subref{Fig:R1}左边的自编码器,\protect\subref{Fig:R2}右边的自编码器} %注意须使用\protect\subref{}进行标号引用
    \label{Fig:RecAccuracy} % 整个组图的标签名
\end{figure}

\section{公式与算法表示}

\subsection{例子:基于主成分分析}

\subsubsection{主成分分析算法}

下面对主成分分析进行介绍。

主成分分析是一种简单的机器学习算法,其功能可以从两方面解释:一方面可以认为它提供了一种压缩数据的方式,另一方面也可以认为它是一种学习数据表示的无监督学习算法。\cite{Goodfellow2016DeepLearning}
通过PCA,我们可以得到一个恰当的超平面及一个投影矩阵,通过投影矩阵,样本点将被投影在这一超平面上,且满足最大可分性(投影后样本点的方差最大化),直观上讲,也就是能尽可能分开。

对中心化后的样本点集$\bm{X}=\{\bm{x}_1,\bm{x}_2,\ldots,\bm{x}_i,\ldots,\bm{x}_m\}$(有$\sum_{i=1}^{m}\bm{x}_i = 0$),考虑将其最大可分地投影到新坐标系\ $\bm{W}= \{\bm{w}_1,\bm{w}_2,\ldots,\bm{w}_i,\ldots,\bm{w}_d\} $,其中$\bm{w}_i$是标准正交基向量,满足$\|\bm{w}_i\|_2 = 1$, $\bm{w}_i^T\bm{w}_j = 0$($i \not= j$)。假设我们需要$d^\prime$($d^\prime < d$)个主成分,那么样本点$\bm{x}_i$在低维坐标系中的投影是$\bm{z}_i = (z_{i1};z_{i2};\ldots;z_{id^\prime})$,其中$z_{ij} = \bm{w}_j^\mathrm{T}\bm{x}_i$,是$\bm{x}_i$在低维坐标系下第$j$维的坐标。
对整个样本集,投影后样本点的方差是
\begin{equation}
\begin{aligned}
    & \frac{1}{m}\sum_{i=1}^m \bm{z}_i^\mathrm{T}\bm{z}_i \\
= & \frac{1}{m}\sum_{i=1}^m (\bm{x}_i^\mathrm{T}\bm{W})^\mathrm{T}(\bm{x}_i^\mathrm{T}\bm{W}) \\
= & \frac{1}{m}\sum_{i=1}^m \bm{W}^\mathrm{T}\bm{x}_i\bm{x}_i^\mathrm{T}\bm{W} \\
= & \frac{1}{m} \bm{W}^\mathrm{T}\bm{X}\bm{X}^\mathrm{T}\bm{W} \\
\end{aligned}
\end{equation}

由于我们知道新坐标系$\bm{W}$的列向量是标准正交基向量,且样本点集$\bm{X}$已经过中心化,则PCA的优化目标可以写为
\begin{equation}
\label{PCA_goal_TMP}
\begin{aligned}
& \max_{\substack{\bm{W}}}  &  tr(\bm{W}^\mathrm{T}\bm{X}\bm{X}^ \mathrm{T}\bm{W}) \\
& \operatorname{ s.t. }  &  \bm{W}^\mathrm{T}\bm{W} = \bm{I} \\
\end{aligned}
\end{equation}

由于$\bm{X}\bm{X}^ \mathrm{ T }$是协方差矩阵,那么只需对它做特征值分解,即
\begin{equation}
\label{PCA_eigenvalue}
\bm{X}^ \mathrm{ T }\bm{X} = \bm{W}\bm{\Lambda}\bm{W}^ \mathrm{ T } \\
\end{equation}
其中$\bm{\Lambda}=diag(\bm{\lambda})$,$\bm{\lambda} = \{\lambda_1,\lambda_2,\ldots,\lambda_m\}$。

具体地,考虑到它是半正定矩阵的二次型,存在最大值,可对\eqref{PCA_goal_TMP}使用拉格朗日乘数法
\begin{equation}
\bm{X}\bm{X}^ \mathrm{ T }\bm{w}_i  = \lambda_i \bm{w}_i \\
\end{equation}

之后将求得的特征值降序排列,取前$d^\prime$个特征值对应的特征向量组成所需的投影矩阵$\bm{W}^\prime =(\bm{w}_1,\bm{w}_2,\ldots,\bm{w}_{d^\prime})$,即可得到PCA的解。PCA算法的描述如算法\ref{PCA_algorithm}所示。

\begin{algorithm} 
	\begin{spacing}{1.3}
		\floatname{algorithm}{算法}
		\caption{主成分分析(PCA)} 
		\label{PCA_algorithm}
		\renewcommand{\algorithmicrequire}{\textbf{输入:}}
		\renewcommand{\algorithmicensure}{\textbf{输出:}} 
		\begin{algorithmic}[1] 
			\Require 样本集$\bm{x}=\{\bm{x}_1,\bm{x}_2,\ldots,\bm{x}_i,\ldots,\bm{x}_m\}$,低维空间维数$d^\prime$ 
			\Ensure 投影矩阵  $\bm{W}^\prime =(\bm{w}_1,\bm{w}_2,\ldots,\bm{w}_{d^\prime})$
			\State 对所有样本中心化$\bm{x}_i \gets \bm{x}_i - \frac{1}{m}\sum_{i=1}^m \bm{x}_i$
			\State  计算样本的协方差$\bm{X}\bm{X}^ \mathrm{T}$
			\State 对协方差矩阵$\bm{X}\bm{X}^ \mathrm{T}$做特征值分解
			\State 取最大的$d^\prime$个特征值所对应的特征向量$\bm{w}_1,\bm{w}_2,\ldots,\bm{w}_{d^\prime}$
		\end{algorithmic}  
	\end{spacing}
\end{algorithm}

\subsubsection{主成分分析可信度评估方法}
记待判定微博$\bm{w}_0$的经典特征向量为$\bm{f}^{c}_{0}$,它的发布者在$\bm{w_0}$前发布的$k$条微博为$\bm{W} = \bm{w}_1,\bm{w}_2,\ldots,\bm{w}_k$,这$k$条微博对应的经典特征向量集为$\bm{F}^{c}_{W} = \{ \bm{f}^{c}_{1},\bm{f}^{c}_{2},\ldots,\bm{f}^{c}_{k} \}$。令$label = 1$代表谣言,$label = 0$代表非谣言。算法的具体流程如算法\ref{PCA_model}所示。

\begin{algorithm}
	\begin{spacing}{1.3}
		\floatname{algorithm}{算法}
		\caption{基于PCA的信息可信度评估} 
		\label{PCA_model}
		\renewcommand{\algorithmicrequire}{\textbf{输入:}}
		\renewcommand{\algorithmicensure}{\textbf{输出:}} 
			\begin{algorithmic}[1] 
				\Require $\bm{f}^{c}_{0}$,$\bm{F}^{c}_{W}$,保留主成分数$n$
				\Ensure 标签$label\in \{0,1\}$
				\State 对所有特征向量应用PCA,保留前$n$个主成分$\bm{o}^{c}_{i} \gets PCA(\bm{f}^{c}_{i}, n)$($i = 0,1,\ldots,k$)
				\State 计算$\bm{F}^{c}_{W}$中各向量的平均距离$\mu$和标准差$\sigma$
				\State 计算阈值$thr = {\mu} / {\sigma}$
				\If {$\min_{1<j\le k} \|\bm{o}^{c}_{0} - \bm{o}^{c}_{j} \|_2 > thr$}
					\State $ label \gets 1 $
				\Else
					\State $ label \gets 0 $
				\EndIf
			\end{algorithmic}
	\end{spacing}
\end{algorithm}

\section{代码表示}

%据悉以下语言被lstlisting支持:Awk, bash, Basi4, C#, C++, C, Delphi, erlang, Fortran, GCL, Haskell, HTML, Java, JVMIS, Lisp, Logo, Lua, make, Mathematica, Matlab, Objective C , Octave, Pascal, Perl, PHP, Prolog,  Python, R, Ruby, SAS, Scilab, sh, SHELXL, Simula, SQL, tcl, TeX, VBScript, Verilog, VHDL, XML, XSLT
%遗憾的是,JavaScript不被支持,请上网搜索支持该语言的方法

\subsection{直接书写代码在.tex中}
下面的代码\ref{plus}是用Python编写的加法函数。

\begin{lstlisting}[language=Python, caption=加法, label=plus, tabsize=2]  
def plusFunc(a, b):
	return a + b 
\end{lstlisting}  

\subsection{引用代码文件}
下面的代码\ref{recursion}是用Python文件中引入的倒序打印$x$到$1$的函数,请查看code文件夹。

\lstinputlisting[language=Python, caption=倒序打印数字, label=recursion, tabsize=2]{code/recursion.py}

\section{列表样式}

\subsection{使用圆点作为项目符号}

\begin{itemize}
\item \textbf{第一章为基础模块示例},是的,本章的名字就是基础模块示例,正如你看到这个样子。
\item \textbf{第二章为不存在},是的,其实它不存在。
\end{itemize}

\subsection{使用数字作为项目符号}

\begin{enumerate}
\item \textbf{第一章为基础模块示例},是的,本章的名字就是基础模块示例,正如你看到这个样子。
\item \textbf{第二章为不存在},是的,其实它不存在。
\end{enumerate}

\subsection{句中数字编号列表样式}

\begin{enumerate*}
    \item \textbf{第一章为基础模块示例},是的,本章的名字就是基础模块示例,正如你看到这个样子;
    \item \textbf{第二章为不存在},是的,其实它不存在。
\end{enumerate*}

