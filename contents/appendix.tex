% Appendix
% 附录页面设置,设置图片和公式计数器从开始,设置图片和公式命名样式
\setcounter{figure}{0} 
\renewcommand{\thefigure}{~附-\arabic{figure}~}
\setcounter{equation}{0} 
\renewcommand{\theequation}{~附-\arabic{equation}~}
\setcounter{table}{0} 
\renewcommand{\thetable}{~附-\arabic{table}~}
\setcounter{lstlisting}{0} 
\makeatletter
  \renewcommand \thelstlisting
       {附-\@arabic\c@lstlisting}
\makeatother
% 开始附录

\chapter*{附\qquad{}录}
% phantomsection语句用于在目录生成正确的跳转位置,使用chapter*等命令时需使用
\phantomsection\addcontentsline{toc}{chapter}{附\qquad{}录}

\phantomsection\addcontentsline{toc}{section}{附录1\quad{}缩略语表}
\section*{附录1\quad{}缩略语表}

\begin{bupttable}{基于浏览者行为的特征}{crowdwisdom2}
    \begin{tabular}{l|l|l}
        \hline \textbf{特征} & \textbf{描述} & \textbf{形式与理论范围}\\
        \hline 点赞量 & 微博的点赞数量 & 数值,$\mathbb{N}$ \\
        \hline 评论量 & 微博的评论数量 & 数值,$\mathbb{N}$ \\
        \hline 转发量 & 微博的转发数量 & 数值,$\mathbb{N}$ \\
        \hline
    \end{tabular}
\end{bupttable}

\begin{bupttable}{基于浏览者行为的复杂特征}{complexcrowdwisdom2}
    \begin{tabular}{l|l|l|l}
		\hline
        \multicolumn{1}{c|}{\multirow{2}{*}{\textbf{类别}}} & \multicolumn{1}{c|}{\multirow{2}{*}{\textbf{特征}}} & \multicolumn{2}{c}{\textbf{不知道叫什么的表头}} \\
        \cline{3-4}
         & & \multicolumn{1}{c|}{\textbf{描述}} & \multicolumn{1}{c}{\textbf{形式与理论范围}} \\
		\hline
        \multirow{3}{*}{正常互动} & 点赞量 & 微博的点赞数量 & 数值,$\mathbb{N}$ \\
		\cline{2-4}
         & 评论量 & 微博的评论数量 & 数值,$\mathbb{N}$ \\
		\cline{2-4}
         & 转发量 & 微博的转发数量 & 数值,$\mathbb{N}$ \\
		\hline
        非正常互动 & 羡慕量 & 微博的羡慕数量 & 数值,$\mathbb{N}$ \\
        \hline
    \end{tabular}
\end{bupttable}
\buptfigure[width=0.15\textheight]{pictures/autoencoder}{自编码器结构}{autoencoder}

\begin{lstlisting}[language=Python, caption=减法, label=minus, tabsize=2]  
def minusFunc(a, b):
	return a - b 
\end{lstlisting}  

\begin{equation}
\label{PCA_goal}
\begin{aligned}
\max_{\substack{\bm{W}}}  &  tr(\bm{W}^\mathrm{T}\bm{X}\bm{X}^ \mathrm{T}\bm{W})
\end{aligned}
\end{equation}

\clearpage
\phantomsection\addcontentsline{toc}{section}{附录2\quad{}数学符号}
\section*{附录2\quad{}数学符号}
\begin{center}
	\begin{tabular}{ccc}
		\multicolumn{2}{c}{\textbf{数和数组}} \\
		\\
		$a$ & 标量(整数或实数)\\
		$\bm{a}$ & 向量\\
		$dim()$ & 向量的维数\\
		$\bm{A}$ & 矩阵\\
		$\bm{A}^\mathrm{T}$ & 矩阵$\textbf{A}$的转置\\
		$\bm{I}$ & 单位矩阵(维度依据上下文而定) \\
 		$diag(\bm{a})$ & 对角方阵,其中对角元素由向量$\bm{a}$确定 \\

	\end{tabular}
\end{center}
